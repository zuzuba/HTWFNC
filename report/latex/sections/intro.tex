
\section{Introduction}\label{sec:intro}
In this section we provide a high level overview of the broad impact of neural networks (NNs) across multiple scientific disciplines. We argue about the importance of the performance of forward prediction and finally we discuss previous work that has been done in the field.

%Do not start the introduction with the abstract or a slightly modified
%version. It follows a possible structure of the introduction. 
%Note that the structure can be modified, but the
%%content should be the same. Introduction and abstract should fill at most the first page, better less.
\mypar{Motivation} In recent years we are witnessing an exponential increase in the amount of data available for analysis in almost all scientific disciplines. As a result, there has been a growing interest in machine learning methods to conduct such analysis. Among these, Neural Networks (NNs) are regarded as one of the most promising techniques. 
They have been successfully applied to a wide range of tasks (often outperforming the state of the art) including medical applications \cite{amato_artificial_2013}, image recognition \cite{krizhevsky_imagenet_2012} and robotics \cite{gu_deep_2016}. 

One of the main drawbacks of NNs is that they make use of a high number of parameters. This means that the network requires a lot of memory resources for storage and a lot of computational resources for training and forward prediction. While the training phase usually is performed on powerful parallel computing architectures where there is an abundance of both computational and memory resources, the platforms that make use of the trained network usually have much more limited capabilities (e.g. mobile phones). This problem has steered attention of the research community toward reducing the memory and computation requirements for trained NNs.

A promising direction along this line of research is known as quantized neural networks (QNNs). The central idea to QNNs is to compress the parameters of the network from their float representation to a light-weight one based on quantization bins. The parameter space is divided into a predefined number of bins and each parameter value 
 The first task is to motivate what you do.  You can
start general and zoom in one the specific problem you consider.  In
the process you should have explained to the reader: what you are doing,
why you are doing, why it is important (order is usually reversed).

For example, if my result is the fastest DFT implementation ever, one
could roughly go as follows. First explain why the DFT is important
(used everywhere with a few examples) and why performance matters (large datasets,
realtime). Then explain that fast implementations are very hard and
expensive to get (memory hierarchy, vector, parallel). 

Now you state what you do in this paper. In our example: 
presenting a DFT implementation that is
faster for some sizes than all the other ones.

\mypar{Related work} Next, you have to give a brief overview of
related work. For a paper like this, anywhere between 2 and 8
references. Briefly explain what they do. In the end contrast to what
you do to make now precisely clear what your contribution is.
