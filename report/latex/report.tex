% IEEE standard conference template; to be used with:
%   spconf.sty  - LaTeX style file, and
%   IEEEbib.bst - IEEE bibliography style file.
% --------------------------------------------------------------------------

\documentclass[letterpaper]{article}
\usepackage[hidelinks]{hyperref}
\usepackage{spconf,amsmath,amssymb,graphicx}

% Example definitions.
% --------------------
% nice symbols for real and complex numbers
\newcommand{\R}[0]{\mathbb{R}}
\newcommand{\C}[0]{\mathbb{C}}

% bold paragraph titles
\newcommand{\mypar}[1]{{\bf #1.}}

% Title.
% ------
\title{A Descriptive Title, not too general, not too long}
%
% Single address.
% ---------------
\name{Markus P\"uschel\thanks{The author thanks Jelena Kovacevic. This paper
is a modified version of the template she used in her class.}} 
\address{Department of Computer Science\\ ETH Z\"urich\\Z\"urich, Switzerland}

% For example:
% ------------
%\address{School\\
%		 Department\\
%		 Address}
%
% Two addresses (uncomment and modify for two-address case).
% ----------------------------------------------------------
%\twoauthors
%  {A. Author-one, B. Author-two\sthanks{Thanks to XYZ agency for funding.}}
%		 {School A-B\\
%		 Department A-B\\
%		 Address A-B}
%  {C. Author-three, D. Author-four\sthanks{The fourth author performed the work
%		 while at ...}}
%		 {School C-D\\
%		 Department C-D\\
%		 Address C-D}
%

\begin{document}
%\ninept
%
\maketitle
%

\begin{abstract}
Neural Networks are a class of models for making inference over complex non-linear functions that have established the state of the art for several machine learning tasks. Despite their success, their diffusion in the world of embedded devices is limited by their memory and computational requirements. These requirements stem from the high number of parameters,  usually stored as floats, that is required to represent a Neural Network. To overcome this, quantization compresses these parameters to a representation that uses a predefined number of bits, $k$. The benefits in terms of memory are evident. However, the computational benefits of this representation have not been explored extensively. In this work we introduce an optimized implementation of a 4-bits quantized neural network. Such level of compression is challenging because byte addressability of computer memories forces even our vanilla implementation to work on multiple data simultaneously. Our optimizations concern ops count reduction,  memory optimization and vectorization. Our implementation is able to achieve a speed-up  up to $\times14$ for some functions and a $\times2.5$ overall speed-up. 
%Neural Networks are a Machine Learning Model where a non-linear function
%is applied to subsequent applications of matrix multiplication and vector addition.
%In order to achieve state of the art results, a vast number are required for the model.
%Generally, these parameters are encoded as either 32-bit or 16-bit 
%floating point numbers, which lead to large memory requirements to store the parameters
%of a trained model. Quantization is a method to compress these parameters to
%an arbitrary fixed precision, while controlling the degradation of that model's precision.
%We achieve this compression by narrowing the range of values in a single parameter matrix to
%the minimum and maximum of the parameters in the matrix and subsequently encoding the values in the
%parameter matrix according to a simple encoding scheme for the values in the narrowed range.
%Once the values have been encoded, we are able to perform matrix multiplication on
%the compressed values rather than needing to perform decompression at each step. The result is that
%given a fixed bit-length for each of the compressed paramters, we are able to generate an
%encoded parameter matrix where each parameter is encoded with that fixed bit-length while ensuring
%only minute model degradation with significant memory savings.

%Describe in concise words what you do, why you do it (not necessarily
%in this order), and the main result.  The abstract has to be
%self-contained and readable for a person in the general area. You
%should write the abstract last.
%Describe in concise words what you do, why you do it (not necessarily
%in this order), and the main result.  The abstract has to be
%self-contained and readable for a person in the general area. You
%should write the abstract last.
%Describe in concise words what you do, why you do it (not necessarily
%in this order), and the main result.  The abstract has to be
%self-contained and readable for a person in the general area. You
%should write the abstract last.
%Describe in concise words what you do, why you do it (not necessarily
%in this order), and the main result.  The abstract has to be
%self-contained and readable for a person in the general area. You
%should write the abstract last.
%Describe in concise words what you do, why you do it (not necessarily
%in this order), and the main result. 

\end{abstract}


\section{Introduction}\label{sec:intro}
In this section we provide a high level overview of the broad impact of neural networks (NNs) across multiple scientific disciplines. We argue about the importance of the performance of forward prediction and finally we discuss previous work that has been done in the field.

%Do not start the introduction with the abstract or a slightly modified
%version. It follows a possible structure of the introduction. 
%Note that the structure can be modified, but the
%%content should be the same. Introduction and abstract should fill at most the first page, better less.
\mypar{Motivation} In recent years we are witnessing an exponential increase in the amount of data available for analysis in almost all scientific disciplines. As a result, there has been a growing interest in machine learning methods to conduct such analysis. Among these, Neural Networks (NNs) are regarded as one of the most promising techniques. 
They have been successfully applied to a wide range of tasks (often outperforming the state of the art) including medical applications \cite{amato_artificial_2013}, image recognition \cite{krizhevsky_imagenet_2012} and robotics \cite{gu_deep_2016}. 

One of the main drawbacks of NNs is that they make use of a high number of parameters. This means that the network requires a lot of memory resources for storage and a lot of computational resources for training and forward prediction. While the training phase usually is performed on powerful parallel computing architectures where there is an abundance of both computational and memory resources, the platforms that make use of the trained network usually have much more limited capabilities (e.g. mobile phones). This problem has steered attention of the research community toward reducing the memory and computation requirements for trained NNs.

A promising direction along this line of research is known as quantized neural networks (QNNs). The central idea to QNNs is to compress the parameters of the network from their float representation to a light-weight one based on quantization bins. The parameter space is divided into a predefined number of bins and each parameter float value is mapped to a bin. The number of bins trades-off the accuracy versus the gain in memory and computation requirements. By ordering the bins, there is a convenient bijective relation between them and the natural numbers. This allows us to represent the approximation of the initial four byte floats with an integer that requires $ceil(\log_2({num~of~bins}))$ bits. This leads to an obvious improvement in memory requirements as long as the number of bins is smaller than $2^{32}$. Furthermore, by fitting more operands in AVX registers and by exploiting spatial locality in caches, quantization results in reduced computational cost for forward prediction.

In this work we present an optimized implementation of a QNN for the forward prediction on the MNIST data set that makes use of fours bits quantization. While this level of quantization can yield substantial improvements in memory and computation requirements, it presents implementation challenges due to the byte addressability of memories and to the lack of built-in data type for four bits integers.

% The first task is to motivate what you do.  You can
%start general and zoom in one the specific problem you consider.  In
%the process you should have explained to the reader: what you are doing,
%why you are doing, why it is important (order is usually reversed).
%
%For example, if my result is the fastest DFT implementation ever, one
%could roughly go as follows. First explain why the DFT is important
%(used everywhere with a few examples) and why performance matters (large datasets,
%realtime). Then explain that fast implementations are very hard and
%expensive to get (memory hierarchy, vector, parallel). 
%
%Now you state what you do in this paper. In our example: 
%presenting a DFT implementation that is
%faster for some sizes than all the other ones.

\mypar{Related work} Next, you have to give a brief overview of
related work. For a paper like this, anywhere between 2 and 8
references. Briefly explain what they do. In the end contrast to what
you do to make now precisely clear what your contribution is.

\section{Background: NNs and QNNs}\label{sec:background}

In this section we formally introduce NNs and QNNs and relative notation.

\mypar{Artificial neural networks}
An artificial neural network (NN) is a non linear map from an input vector $\mathbf{x} \ \in \mathbb{R}^{d_i}$ to an output vector $f(\mathbf{x} ) = \mathbf{y} \ \in \mathbb{R}^{d_o}$. The map $f$ is built recursively applying at step $t$ a linear transformation $ \mathbf{a}_t = \mathbf{W}_t\mathbf{x}_t + \mathbf{b}_t$ and a non-linear transformation $\mathbf{x}_{t+1} = \phi_t(\mathbf{a}_t)$. The matrix $\mathbf{W}_t$ is called \emph{weight matrix}, and the vector $\mathbf{b}_t$ is called \emph{bias vector}. The step $t$ is also know as the \emph{layer} index. 

\mypar{Quantized neural network}
A quantized neural network (QNN) is a NN that uses low precision weight matrix and bias vector. Formally, given a NN with parameters $\{\mathbf{W}_t\}, \{\mathbf{b}_t\}$ and activation functions $\phi_t$, the quantized implementation is the NN that apply at each layer the linear transformation $\mathbf{a}_t = \mathcal{Q}(\mathbf{W}_t) \mathcal{Q}(\mathbf{x}_t) + \mathcal{Q}(\mathbf{b}_t)$ and a non-linear transformation $\mathbf{x}_{t+1} = \phi_t(\mathbf{a}_t)$, where the function $\mathcal{Q}(\cdot)$ is introduced in the following.

\mypar{Matrix quantization} For a matrix $\mathbf{A}$, the function $\mathcal{Q}(\mathbf{A})$ returns a low precision encoding of the matrix $\mathbf{A}$. It first computes the minimum entry ($mn$) and the maximum entry ($mx$) of the matrix $\mathbf{A}$. Then, given $k$ bits it builds a linear binning of the continuous interval $[mn,mx]$ into $2^k$ bins. The bin size of the quantization $\Delta(\mathbf{A})$ is then \[\Delta(\mathbf{A}) = \frac{mx - mn}{2^k}\] To insure that the value $0$ is represented exactly as a bin value, its index is computed as \[z(\mathbf{A}) = sat([-mn/\Delta(\mathbf{A})])\] where the brackets $[\cdot]$ stand for the rounding to the closest integer and the $sat(\cdot)$ function saturates an integer value into the integer value representable with k bits, hence it reads $sat(n) = \max(0, \min(n,2^k) )$. The bin values are then $\{ (i-z(\mathbf{A})) \Delta(\mathbf{A}), \ \ i = 0, \dots, 2^k -1 \}$. Then every entry $A_{ij}$ is quantized to the closest bin value. The quantize matrix  $\mathcal{Q}(\mathbf{A})$ and the quantized integer matrix $\tilde{\mathcal{Q}}(\mathbf{A})$ have respectively the bin value and the bin index as their $ij$ entry. Note that the matrix $\mathcal{Q}(\mathbf{A})$ is a real-valued matrix, while $\tilde{\mathcal{Q}}(\mathbf{A})$ is k-bit integer valued, and also that the following holds:
\begin{equation}\label{equation:affine_transf}
\mathcal{Q}(\mathbf{A}) = (\tilde{\mathcal{Q}}(\mathbf{A}) -z(\mathbf{A}) \mathbf{J}  ) \Delta(\mathbf{A})
\end{equation} 
where $\mathbf{J}$ is a matrix with all entries equal to one. The algorithm to compute $\tilde{\mathcal{Q}}(\mathbf{A})$  is showed in \cref{algorithm:quantize}.

\begin{algorithm}
	\caption{Quantize}\label{algorithm:quantize}
	\begin{algorithmic}[1]
		\State compute $mn = \min A_{ij}$ and $mx = \max A_{ij}$
		\State $\Delta = \frac{mx - mn}{2^k}$.
		\State $z = -mn/\Delta$
		\For{$i,j = 1, \dots N$}
			\State $\tilde{\mathcal{Q}}(\mathbf{A})_{ij} = saturate([A_{ij}/\Delta + z ])$ 
		\EndFor
	\end{algorithmic}
\end{algorithm}

\mypar{Quantized Matrix-Matrix Multiplication} Given two matrices $\mathbf{L}$ and $\mathbf{R}$, we want to compute the product $\mathcal{Q}(\mathbf{L}) \mathcal{Q}(\mathbf{R})$. Using \cref{equation:affine_transf} and we write the product as 
\begin{align}\label{equation:qmmm}
\begin{split}
& \mathcal{Q}(\mathbf{L}) \mathcal{Q}(\mathbf{R}) =\\
 & \Delta (\mathbf{L})\left( \tilde{\mathcal{Q}}(\mathbf{L}) - z(\mathbf{L})\mathbf{J} \right)
\left( \tilde{\mathcal{Q}}(\mathbf{R}) - z(\mathbf{R})\mathbf{J} \right)  \Delta (\mathbf{R})
\end{split}
\end{align}
Inverting the equation \ref{equation:affine_transf}  we can then obtain the k-bit integer valued product matrix as 
\begin{equation}\label{equation:affine_inverse}
\tilde{\mathcal{Q}}(\mathbf{LR}) = sat([\frac{1}{\Delta(\mathbf{LR})}\mathcal{Q}(\mathbf{L}) \mathcal{Q}(\mathbf{R}) + z(\mathbf{LR}) \mathbf{J}])
\end{equation} The algorithm is showed in \cref{algorithm:qmmm}.

\begin{algorithm}
	\caption{QMMM}\label{algorithm:qmmm}
	\begin{algorithmic}[1]
		\State compute $\mathcal{Q}(\mathbf{L}) \mathcal{Q}(\mathbf{R})$ as in \cref{equation:qmmm}
		\State compute the k-bit integer matrix $\tilde{\mathcal{Q}}(\mathbf{LR})$ as in \cref{equation:affine_inverse}
 	\end{algorithmic}
\end{algorithm}


\section{Your Proposed Method}\label{sec:yourmethod}

\mypar{Baseline implementation}

Our baseline implementation consists of two parts:
\begin{itemize}
\item Scalar quantization
\item Scalar matrix to matrix multiplication
\end{itemize}

\textbf{Scalar Quantization: } Our input data \emph{v} is a vector of 32-bit floats that contains the values for an arbitrary matrix of size $\emph{rows} \times \emph{columns} \in \mathbb{N} \times \mathbb{N} $, i.e.\ $ \emph{v} \in \mathbb{R}^{ \emph{rows} \cdot \emph{columns} } \cong \mathbb{R}^{ \emph{rows} \times \emph{columns} } $. We traverse the values in \emph{v} to identify the minimum (\emph{min}) and maximum (\emph{max}) of the vector, i.e. $ \emph{min} = \min_{ i \in \emph{rows} \cdot \emph{columns} } v[i] , \emph{max} = \max_{i \in \emph{rows} \cdot \emph{columns} } v[i] $. Subsequently, given a desired precision $ \emph{p} \in \mathbb{N} $, we compute the size of a cell in our range $[\emph{min},\emph{max}] \subset \mathbb{R}$ via $ \Delta = \frac{ \emph{max} - \emph{min} }{ 2^{\emph{p}}} \in \mathbb{R} $. Finally, due to the importance of the 0 value in Neural Networks, we apply a correction value to ensure that the 0 value is encoded exactly, computed via $ \zeta = \frac{ | min | }{ \Delta } - \lfloor \frac{ | min | }{ \Delta } \rfloor $. Finally, we may encode every value in \emph{v} according to the encoding scheme: $ \emph{q}[i] = \frac{ \emph{v}[i] } {\Delta} + \zeta , i \in [0, \emph{rows}\cdot\emph{columns}] \cap \mathbb{N} $.

\textbf{Scalar matrix to matrix multiplication: } Once the parameter matrix has been quantized, we arrive at a quantized vector $ \emph{q} \in  \left([0, 2^{\emph{p}} - 1]\right)^{ \emph{rows} \cdot \emph{columns} }$. Assuming that our input vectors ($ \emph{x} \in \mathbb{R}^{rows} $), we must now compute $ \emph{q} \emph{x} $. Given that $ \emph{q} = ( \frac{ \emph{v}[i] }{ \Delta} + \zeta )_{i \in \emph{rows}\cdot\emph{columns}} \cdot \emph{x} = \frac{1}{\Delta} \left(  \emph[v] \cdot \emph{x} \right) + \zeta \cdot (1,1,1,...,1,1) $. Moreover, assuming that the input vectors are encoded in compressed format ($\widetilde{{x}}$), we may rearrange the terms as follows: 
$ \widetilde{{x}}^{T} \cdot \emph{q} = \frac	{1}{\Delta_q \cdot \Delta_x} \left( \emph{x}' \cdot \emph{v} ) + \frac{\zeta_q}{\Delta_x} \emph{x} + \frac{\zeta_x}{\Delta_q} \emph{v} + \zeta_q \cdot \zeta_x \cdot (1,1,1,...,1,1) \right)$.

%Now comes the ``beef'' of the paper, where you explain what you
%did. Again, organize it in paragraphs with titles. As in every section
%you start with a very brief overview of the section.
%
%For this class, explain all the optimizations you performed. This mean, you first very briefly
%explain the baseline implementation, then go through locality and other optimizations, and finally SSE (every project will be slightly different of course). Show or mention relevant analysis or assumptions. A few examples: 1) Profiling may lead you to optimize one part first; 2) bandwidth plus data transfer analysis may show that it is memory bound; 3) it may be too hard to implement the algorithm in full generality: make assumptions and state them (e.g., we assume $n$ is divisible by 4; or, we consider only one type of input image); 4) explain how certain data accesses have poor locality. Generally, any type of analysis adds value to your work.
%
%As important as the final results is to show that you took a structured, organized approach to the optimization and that you explain why you did what you did.
%
%
%Mention and cite any external resources including library or other code.
%
%Good visuals or even brief code snippets to illustrate what you did are good. Pasting large amounts of code to fill the space is not good.

\section{Experimental Results}\label{sec:exp}
\graphicspath{{../../plots/}}
\epstopdfsetup{outdir=./figures/}

In this section we evaluate empirically the optimizations outlined in \cref{sec:yourmethod}. Every code version is tested for correctness on a small size example with hand-computed output and on several large-size random instances with output given by the naive implementation.

\mypar{Experimental setup}
For the empirical evaluation of the code, we use a Skylake processor (3.5 GHz, L1 cache 128 KB, L2 cache 1 MB, L3 cache 6 Mb). The compiler used is g++ with flags "-O3 -fno-tree-vectorize -march=native -mavx". The matrix size varies in the range $[30,1000]$.

\mypar{Results: operational count optimization}
We first evaluate the gain the in runtime due to the optimization in the operational count performed with the trick as in \cref{equation:qmmm_smart}. From \cref{figure:performance_qmm_kernel} we can see the decrease in performance of the trick version with respect to the naive implementation. \Cref{figure:cycles_qmm_comparison} shows that the operation count optimization gives an overall speed-up of $15 \%$ (this is the difference in run time between \emph{naive} and \emph{naive\_trick}). In \cref{figure:Cycles_trick} we can see the contribution of each function of the vanilla implementation of the pipeline to the overall runtime. In the following the trick version is further optimized.

\begin{figure}[h]
\includegraphics[width=0.5\textwidth]{Performance_qmm.eps}
\caption{Performance plot for the QMM kernel}
\label{figure:performance_qmm_kernel}
\end{figure}

\begin{figure}[h]
\includegraphics[width=0.5\textwidth]{Cycles_qmm_comparison.eps}
\caption{Runtime plot of the overall pipeline}
\label{figure:cycles_qmm_comparison}
\end{figure}

\begin{figure}[h]
\includegraphics[width=0.5\textwidth]{Cycles_trick.eps}
\caption{Contribution of each naive-version sub-function in the overall runtime.} 
\label{figure:Cycles_trick}
\end{figure}


\mypar{Results: blocking for MMM}
The blocking parameter used is $N_b = 30$ for cache blocking  and $n_b = 3$ for register blocking, as described in \cref{sec:yourmethod}. \Cref{figure:performance_qmm_kernel} shows the gain in performance with respect to the \emph{naive\_trick} implementation. The performance gain for large  large $n$ is approximately $2$X. Note that the blocking parameter for cache and for register are not fine tuned, so a further speed-up could be possible.


\mypar{Results: vectorization} 
In this paragraph we evaluate the performance gain thanks to vectorization for \emph{trick\_vector}, \emph{quantize}, \emph{add\_trick\_vector} and \emph{round\_saturation} as outlined in \cref{sec:yourmethod}. \Cref{figure:performance_add_vector} and \cref{figure:performance_trick_vector} show a linear increase of the performance for small instances. This is due to a border effect. The instance size in the  performance plot are not in general a multiple of the vector size ($16$X$16 \ Bits$), hence for small instance size the scalar computation is be non-negligible. Moreover the scalar contribution decreases linearly with $n$.

\begin{figure}[h]
\includegraphics[width=0.5\textwidth]{Performance_add_vector.eps}
\caption{Performance plot of the function \emph{add\_vector}}
\label{figure:performance_add_vector}
\end{figure}

\begin{figure}[h]
	\includegraphics[width=0.5\textwidth]{Performance_quantize.eps}
	\caption{Performance plot of the function \emph{quantize}}
	\label{figure:performance_quantize}
\end{figure}

\begin{figure}[h]
\includegraphics[width=0.5\textwidth]{Performance_trick_vector.eps}
\caption{Performance plot of the function \emph{trick\_vector}}
\label{figure:performance_trick_vector}
\end{figure}

\begin{figure}[h]
\includegraphics[width=0.5\textwidth]{Performance_round_saturation.eps}
\caption{Performance plot of the function \emph{round\_saturation}}
\label{figure:performance_round_saturation}
\end{figure}

The maximal gain in performance that the vectorization can allow for the functions \emph{trick\_vector} and \emph{add\_trick\_vector} is $16$X, that is the size of the accumulator vector for 16 bit integers. The measured performance gain are respectively  $9.8$X and $8.5$X.

\Cref{figure:performance_quantize} and \cref{figure:performance_round_saturation} show the performance plot for the functions \emph{quantize} and \emph{round\_saturation}. The measured performance gain are respectively $9.2$X and $14.2$X. 

In \cref{figure:cycles_qmm_comparison} we can see the runtime plot for the whole pipeline. The implementation \emph{trick\_blocking\_AVX} is obtained combining the \emph{QMM\_kernel\_blocking} with the vectorized implementation of all the other sub-functions. The overall speedup is $2.5$X.
\section{Conclusions}
We presented an optimized implementation of 4-bits quantization and 4-bits quantized forward prediction for Neural Networks. Our experimental results show the benefits of performing vectorization and blocking to achieve significant speed-ups over a naive implementation. The key ingredient in our approach lies in the design of the data structure $\emph{uint4x4\_t}.$ In particular it allows us to easily transfer optimization techniques for standard float MMM to QMMM, such as blocking.
Moreover, we showed that, despite the fact that the bulk of the computation happens within the kernel, vectorizing functions that support it can yield a substantial improvement.

%decision to map a tile of $2 \times 2$ floats to a single 16-bit integer (using a struct of 4 members with 4 reserved bits respectively). Through this construction, the algorithms outlined in the previous are accessible in a convenient form that lends itself to simply vectorize the load, quantize, round and saturate functions, while also enabling a simplified blocking scheme for quantized matrix multiplication.

\section{Further comments}

Here we provide some further tips.

\mypar{Further general guidelines}

\begin{itemize}
\item For short papers, to save space, I use paragraph titles instead of
subsections, as shown in the introduction.

\item It is generally a good idea to break sections into such smaller
units for readability and since it helps you to (visually) structure the story.

\item The above section titles should be adapted to more precisely
reflect what you do.

\item Each section should be started with a very
short summary of what the reader can expect in this section. Nothing
more awkward as when the story starts and one does not know what the
direction is or the goal.

\item Make sure you define every acronym you use, no matter how
convinced you are the reader knows it.

\item Always spell-check before you submit (to me in this case).

\item Be picky. When writing a paper you should always strive for very
high quality. Many people may read it and the quality makes a big difference.
In this class, the quality is part of the grade.

\item Conversion to pdf (latex users only): 

dvips -o conference.ps -t letter -Ppdf -G0 conference.dvi

and then

ps2pdf conference.ps
\end{itemize}

\mypar{Graphics} For plots that are not images {\em never} generate (even as intermediate step)
jpeg, gif, bmp, tif. Use eps, which means encapsulate postscript, os pdf. This way it is
scalable since it is a vector graphic description of your graph. E.g.,
from Matlab, you can export to eps or pdf.

Here is an example of how to get a plot into latex
(Fig.~\ref{fftperf}). Note that the text should not be any smaller than shown.

\begin{figure}\centering
  \includegraphics[scale=0.33]{dft-performance.eps}
  \caption{Performance of four single precision implementations of the
  discrete Fourier transform. The operations count is roughly the
  same. {\em The labels in this plot are too small.}\label{fftperf}}
\end{figure}



% References should be produced using the bibtex program from suitable
% BiBTeX files (here: bibl_conf). The IEEEbib.bst bibliography
% style file from IEEE produces unsorted bibliography list.
% -------------------------------------------------------------------------
\bibliographystyle{IEEEbib}
\bibliography{HTWFNC_library.bib}

\end{document}

